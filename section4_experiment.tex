\section{Experimental Results}
\label{sec:experiment}



\subsection{Graph Triplet}
We use MNIST dataset \cite{LecunIEEE1998} to build the graph triplet, and use SplineCNN layer \cite{FeyCVPR2018splinecnn} \cite{FeyICLR2020DGMC} to replace the CNN layer which is normally used in computer vision task. Each MNIST image is represented as a graph, whose nodes correponds to the pixels in the image and the edge is formed by connected each pixel to all its neighboring pixels. Therefore, based on this setup, each MNIST image is a graph with the same number of nodes, which is $784(=28^2)$. The graph triplet is formulated as followes: for each sample as the anchor in the training dataset, we randomly select another sample which belongs to the same class as the positive sample, and randomly select one sample from the rest classes as the negative sample. 

Similar to CNN neural network configuration, we build a SplineCNN network, which contains two SplineCNN layers, and ech SplineCNN layer is followed by an exponentail linear unit (ELU) and a maxpooling layer. There are two fully connected layers, the first fully connected layer is also followed by an ELU layer, the output dimension of the final fully connected layer is $10$ (we also choose the final dimension to be $2$ for visualization). The initial learning rate is $10^{-3}$. Stochastic gradient descent with Adam method is used for optimization and the total training epochs is $20$.

For evaluation, since the total number of samples of MNIST test dataset is $10,000$ and they are not balanced for each class, we randomly choose 500 samples from each class. We use all-to-all evaluation protocal, which means for each test sample, Euclidean distance is computed against all the rest samples. Therefore, in total, we generate $1,247,500$ genuine scores and $11,250,000$ impostor scores. Since MNIST dataset is a simple dataset with a few number of classes and a large number of samples for each class, it turns out that there is no overlap between genuine scores and impostor scores. The equal error rate(EER) is thus zero.