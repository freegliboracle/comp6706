\section{Introduction}
Fingerprint used for human identification can be dated back to over one hundred years ago when people found that no two individuals shared the same fingerprints. Fingerprint was used in forensics since criminals usually unintentionally left their fingerprints in crime scenes and their fingerprints can be used as evidence against themselves. These applications calls for the use of fingerprint and thus research on fingerprint recognition. Compared with other biometrics such as face, iris and voice, fingerprint is highly distinctive, permanent, easy to accept and easy to collect \cite{Maltoni2009}. This makes fingerprint one of the most popular biometrics which is used in a large number of recognition domains such as healthcare, electronic payment, border crossing and forensics.

Because manual fingerprint recognition requires human experts to match fingerprint one by one, the cost of training such experts is high, and matching fingerprint with fingerprints in the database is time consuming and slow, the demand for automatic fingerprint recognition is growing quickly. This makes fingerprint recognition a popular pattern recognition research problem which has attracted significant attention from both the academia and industry. Although many fingerprint recognition solutions have been proposed and showed their effectiveness over the past few decades, fingerprint recognition is still challenging especially for low quality fingerprints due to fingerprint noise and distortion. Therefore, it is an intriguing  research problem even till nowadays, and accurate and fast fingerprint recognition algorithm is in demand.

Generally speaking, the fingerprint images can be classified as 3 categories: plain, latent and rolled \cite{nimkarFingerprintSegmentationAlgorithms2014}.
The plain fingerprint images are usually acquired by touching the fingers on a flat surface.
The latent images are usually collected from some real crime scenes, therefore is usually low quality and contains a lot of noise.
And the rolled fingerprint images are often collected by rolling the fingers from one side to another, which can keep most of the fingerprint ridge information \cite{nimkarFingerprintSegmentationAlgorithms2014}.

There are seven kinds of main factors which is responsible for the intra-class variations: displacement, rotation, partial overlap, non-linear distortion, press and skin condition,  noise and feature extraction errors \cite{Maltoni2009}.
The most important and difficult factor is the non-linear distortion problem. This kind of error is generated on the process of sensing  the 3D fingerprint shapes into a 2D flat surface. Most fingerprint matching algorithms usually do not consider such variations, and consider the obtained fingerprint images as non-distorted by assuming that it was produced by a correct finger placement. Fingerprint recognition can be divided into two parts: fingerprint segmentation and fingerprint matching. Fingerprint segmentation is the process which extract the foreground regions from the original fingerprint images \cite{Maltoni2009}. There are many different kinds of segmentation methods and in this survey and project, we will use segmented images and will only concentrate on the matching part. In the following survey details  part, we will introduce some newly published deep learning-based fingerprint recognition algorithms.

Fingerprint recognition, also called fingerprint matching, can basically be divided into two categories: minutiae based matching and non-minutiae based matching. Minutiae is considered as stable and robust local fingerprint ridge characteristics. Minutiae is usually represented by location, orientation, and minutiae type (i.e., ridge endings and ridge bifurcations). Therefore, minutiae based matching methods try to align two fingerprints by matching the largest number of corresponding minutiae pairs. For low quality fingerprint when it is difficult to extract minutiae, non-minutiae based matching comes into play. Some other fingerprint features such as ridge orientation, ridge frequency,  and ridge shape. To date, most fingerprint recognition algorithms are based on minutiae because of its stability and robustness.

Conventional minutiae based fingerprint recognition method considers matching minutiae pairs by comparing minutiae between two fingerprints. The fast development and increasing popularity of deep learning methods shows remarkable progress in computer vision tasks. Some challenging biometric problems, such as face recognition \cite{SchroffCVPR2015facenet} and iris recognition \cite{ZhaoICCV2017}, have achieved great success thanks to deep learning based methods. There are also some works on using deep learning based methods for fingerprint matching and research shows that deep learning method can also improve fingerprint matching in some scenarios. As another popular research area, graph neural network has also made significant progress over the last decade. Since minutiae of one fingerprint can be formulated as a graph, this implies the possibility of applying graph neural network for matching minutiae graphs. 

This paper is organized as follows: we first discuss and compare related work in Section \ref{sec:related}. Section \ref{sec:method} presents our methods for fingerprint recognition and the corresponding experiment result is shown in Section \ref{sec:experiment}. Conclusion and future work is in Section \ref{sec:conclusion}.