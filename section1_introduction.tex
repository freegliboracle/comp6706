\section{Introduction}
The first time that fingerprint was used for human identification can be traced back to over one hundered years ago \cite{Maltoni2009}. Since people found that no two individuals have the exactly same fingerprints, fingerprint starts to be used in forensics. For example, people who committed crimes might unintentionally leave their fingerprints in the crimes scenes which can later be identified as effective evidence and can help people find the criminals. Thus fingerprints starts to be used as one biometric for human identification and there has been so far a lot of research on human fingerprint recogntion. Compared with other biometrics such as face, iris and palmprint, fingerprint is highly distinctive, permanent, easy to accept and easy to collect \cite{Maltoni2009}, making itself one of the most popular biometrics which is today used in various recognition domains such as healthcare, E-business, border crossing and forensics.

Normally manual fingerprint recognition requires human experts to be invovled so that fingerprint can be matched against other fingerprint one by one. However, the cost of training such experts is high, and matching fingerprint with fingerprints in the database which main contain a huge number of registerd fingerprints is time consuming and slow, so the demand for an automated fingerprint recognition system is growing quickly. As a popular pattern recognition research problem, fingerprint recognition has attracted significant attention of both the academia and the industry. Although many fingerprint recognition algorithms have been proposed by different researchers over the past few decades and those algorithms showed their effectiveness in some fingerprint recognition domain, this is still a quite challenging problem especially for low quality fingerprints which happens often during acquisition due to unintended fingerprint noise and distortion. Therefore, fingerprint recogntion is an intriguing research problem even till today, and an accurate and fast fingerprint recognition algorithm is always preferred.

Generally speaking, the fingerprint images can be classified as 3 categories: plain fingerprint, latent print and rolled fingerprint \cite{nimkarFingerprintSegmentationAlgorithms2014}.
The plain fingerprint images are usually acquired by directly touching the fingers on a flat surface.
The latent images are usually collected from some real crime scenes, therefore these fingerprints are usually of low quality and contain a lot of noise.
And the rolled fingerprint images are often collected by rolling the fingers from one side to another, which can keep most of the fingerprint ridge information \cite{nimkarFingerprintSegmentationAlgorithms2014}.

There are seven kinds of main factors which is responsible for the intra-class variations: displacement, rotation, partial overlap, non-linear distortion, press and skin condition,  noise and feature extraction errors \cite{Maltoni2009}.
The most important factor and also one of the most difficult problems is the non-linear distortion problem. The distortion is usually generated during the process of sensing the 3D fingerprint shapes onto a 2D flat surface. Most fingerprint matching algorithms usually do not directly deal with such issue, and treat the obtained fingerprint images as non-distorted by assuming that it was produced by a correct finger placement. Fingerprint recognition can be divided into two parts: fingerprint segmentation and fingerprint matching. Fingerprint segmentation is the process which extracts the foreground regions from the original fingerprint images \cite{Maltoni2009}. There are many different kinds of segmentation methods, so we will use segmented images and will only concentrate on the matching part.

Fingerprint recognition algorithm, or fingerprint matching algorithm, can basically be divided into two categories: minutiae based matching and non-minutiae based matching. Fingerprint minutiae is considered as stable and robust local fingerprint ridge characteristics. Minutiae is usually represented by location, orientation, and minutiae type (i.e., ridge endings and ridge bifurcations). Therefore, minutiae based matching methods try to align two fingerprints by matching the largest number of corresponding minutiae pairs. For low quality fingerprint images whose minutiae are difficult to extract, non-minutiae based matching should be considered. Non-minuatia based matching usually includes some other fingerprint features such as ridge orientation, ridge frequency, and ridge shape. To date, most fingerprint recognition algorithms are based on minutiae because of its stability and robustness.

Conventional minutiae based fingerprint recognition method considers matching minutiae pairs by comparing minutiae between two fingerprints. The fast development and increasing popularity of deep learning methods shows remarkable progress in computer vision tasks. Some challenging biometric problems, such as face recognition \cite{SchroffCVPR2015facenet} and iris recognition \cite{ZhaoICCV2017}, have achieved great success thanks to deep learning based methods. There are also some works on using deep learning based methods for fingerprint matching and research shows that deep learning method can also improve fingerprint matching in some scenarios. As another popular research area, graph neural network has also made significant progress over the last decade. Since minutiae of one fingerprint can be formulated as a graph, this implies the possibility of applying graph neural network for matching minutiae graphs. 

Our contribution in this work is as follows:
\begin{enumerate}
    \item We propose a way to extract discriminative local fingerprint features, not restricted to fingerprint minutia, that can deal with fingerprint of different qualities. To achieve this target, we use deep learning based method, which shows its superiority in various computer vision tasks in recent years, to extract such feature. This deep neural can not only accurately locate the region of interest, but also learn discriminative local feature representation. 
    \item We propose graph neural network for fingerprint matching, which to our knowledge is the first work of applying graph neural network for fingerprint matching. The first thought is that fingerprint matching performance suffers from rotation, partial overlapping and non-linear distorn issues, and previous fingerprint matching algorithm cannot successfully deal with this issue. However, graph based matching is more robust to such noise. The second thought is although in the past there were some works of building fingerprint graph for matching, the performance is not as good as simple fingerprint pair or triplet matching.
    \item Reserach on graph neural network has made great progress in recent years and has achieved success in many computer vision tasks. We hope our work on fingerprint recognition can make a contribution to the research on both fingerprint recognition and graph neural network.
\end{enumerate}

There are some limitations of our work:
\begin{enumerate}
    \item Since our work is based deep learning based method, which usually requires a lot of data to training the neural network, thus the success of our work relies on the amount of data and the quality of data we have. Deep neural work will show its power once more data is available. However, conventional hard coded matching conditions do not suffer from this problem.
    \item Similar to other maching learning task, our work main also have overfitting issue due to the limited amount of data that we have. Therefore, to mitigate this issue, a large number of experiments on different sources of fingerprint data should be made to validate the generality power of our matching algorithm.
\end{enumerate}

This paper is organized as follows: we first discuss and compare related work on fingerprint recognition in Section \ref{sec:related}. Section \ref{sec:method} presents our methods for fingerprint recognition and the corresponding experiment result is shown in Section \ref{sec:experiment}. Conclusion and future work is given in Section \ref{sec:conclusion}.