\section{Related Work}
\label{sec:related}

\subsection{Conventional Fingerprint Recognition Method}
Minutiae based method is the most widely used method for fingerprint recognition. Consider two fingerprint images with M and N minutiae respectively. Each minutiae is represented by location, orientation and type. A solution is required to match as many minutiae pairs as possible. To achieve this goal, local minutiae structures are formed to match minutiae. Compared with global minutiae matching \cite{ChikkerurICB2006}, local minutiae matching are computationally cheaper and more robust against fingerprint distortion. Earlier approaches can be found in \cite{HrechakPR1990} and \cite{KovacsTPAMI2000}. To describe local minutiae structure, two methods were proposed: nearest neighbor based and fixed radius based. For nearest neighbor based method, the number of neighbors is fixed \cite{JiangICPR2000} so each minutiae has same number of nearest neighbors. For fixed radius method, a radius is given so that all minutiae that fall within the circle defined by the minutiae as the center and the radius are included \cite{CappelliTPAMI2010mcc}. However, both approaches have their own drawbacks. Nearest neighbor method suffers from missing minutiae and spurious minutiae. Although fixed length method alleviates the issue of missing minutiae and spurious minutiae, it has potential border issues, which means that the minutiae near the border of the circle should be properly treated.

\subsection{Deep Learning Method for Fingerprint Recognition}
The past decade has witnessed the great progress of deep learning in computer vision and pattern recognition \cite{HeCVPR2016ResNet} \cite{Simonyan2014VGG} \cite{SzegedyCVPR2015InceptionV1}. The name “deep learning” is derived from the architecture of deep artificial neural networks, which belongs to the family of machine learning methods. Compared with traditional neural networks which have only one or two layers, deep neural network can have tens or hundreds of layers. Such great progress of deep learning has also facilitated not only the research on fingerprint matching \cite{CaoTPAMI2018}, but also the research on other fingerprint recognition related tasks, such as minutia extraction \cite{TangIJCB2017} \cite{NguyenICB2018}. \cite{LinTIFS2018} did the pioneering work on matching with contactless and contact-based 2D fingerprint with convolutional neural network (CNN), while \cite{LinPR2018} also used CNN to match 3D fingerprint, and both works showed that fingerprint recognition with CNN based method can achieve outperforming results. Instead of extracting minutiae for fingerprint matching, \cite{EngelsmaTPAMI2019} trained a deep neural network to extract fixed-length fingerprint descriptor and regarded minutiae extraction as one step to achieve the final fixed-length representation.

\subsection{Graph Neural Network}
Minutiae of one fingerprint can be formulated as a graph by connecting each minutiae based on some predefined rules. However, due to fingerprint noise and distortion, minutiae graph matching can achieve as good performance as other minutiae based matching method \cite{ChikkerurICB2006}. There is another related research area called graph neural network (GNN).The success of deep neural networks has also boosted the research on graph neural network  in recent years \cite{WuTNNLS2020} \cite{ZhouAI2020} \cite{XuICLR2019GIN}. Compared with images that are structured data, there are also unstructured data in the real world, such as social networks, molecules, and citation networks. Deep neural networks have some limitations to model such kind of data, but graph neural networks show their own flexibility. Motivated by CNN, there is a similar graph convolutional networks called GCN \cite{KipfICLR2017}. A graph is represented by nodes and edges. For classification task, it can be divided in to node level based classification and graph level based classification. GNN can also be used for graph matching \cite{ZanfirCVPR2018} \cite{LiICML2019}. Graph matching is an NP-hard problem, and thus requires complex matching algorithm. \cite{SarlinCVPR2020superglue} formulated graph matching as optimal transport problem which can be solved an efficient approximation algorithm called Sinkhorn algorithm \cite{CuturiNIPS2013sinkhorn}.